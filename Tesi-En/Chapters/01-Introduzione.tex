\chapter{Introduction}
\label{introduzione}

\section{Premise and problem positioning}

% ORIGINAL
Contemporary education finds itself at the center of an unprecedented structural crisis. Empirical data converge toward an alarming picture: according to the OECD Education at a Glance 2023 report, over 68\% of fifteen-year-old students in OECD countries show clinically significant levels of academic performance anxiety, representing a 34\% increase compared to the 2015 survey \cite{OECD2023}. Simultaneously, the International Association for the Evaluation of Educational Achievement (IEA) documents that 71\% of teachers report difficulties in managing learning environments characterized by dysfunctional competition and chronic demotivation \cite{IEA2023}.

These quantitative indicators do not represent mere statistical fluctuations, but symptoms of a deeper structural problem: the paradigmatic crisis of the educational model founded on intraspecific selection among individuals. This model, derived from a mechanical and uncritical transposition of Darwinian principles from the biological domain to the pedagogical one, has generated what contemporary sociological literature defines as \textit{educational competitive syndrome} -- a phenomenon in which learning transforms from a collaborative process of knowledge construction into a competitive dynamic, where individual success occurs at the expense of opportunities for shared cognitive growth.

\newpage 
\subsection{The paradox of educational competition}

The paradox of Western educational systems is evident: such systems continue to select individuals through zero-sum competition and cognitive isolation, despite post-industrial society requiring diametrically opposite competencies, such as collaboration, systems thinking, and collective intelligence.

Pierre Bourdieu and Jean-Claude Passeron had already identified this structural contradiction in 1977, defining it as ``symbolic violence masked as meritocracy'' \cite{Bourdieu1977}. The longitudinal research by Duckworth et al., conducted on a sample of 12,847 students followed for eight years, empirically confirms the sociological intuition: students exposed to highly competitive educational systems show an average reduction of 28\% in intrinsic motivation for learning and a 41\% increase in mood disorders related to performance \cite{Duckworth2019}.

But the damage is not only individual -- it is epistemological. Competition among people generates what we can define as \textit{cognitive silos effect}: knowledge becomes private property to be protected rather than a collective resource to be amplified. The result is a systematic impoverishment of the innovative capacity of learning groups, documented by Slavin's meta-analysis of 847 international studies \cite{Slavin2020}.

\section{Theoretical gap and paradigmatic void}

\subsection*{The absence of a unifying framework:}

Systematic analysis of the scientific literature of the last three decades reveals a methodological paradox: while robust empirical evidence exists on the superior effectiveness of collaborative learning compared to traditional competitive models (average effect size: $d = 0.74$, based on 1,247 studies), there lacks a systematic theoretical framework capable of explaining \textit{why} collaboration works and \textit{how} to optimally design it \cite{Johnson2021}.

Most innovative proposals in the field of educational technology remain fragmentary, focusing on specific techniques (cooperative learning, problem-based learning, peer instruction) rather than on paradigmatic transformations. As Sawyer astutely observes: ``We have a collection of best practices without a unifying theory to connect them'' \cite{Sawyer2022}.

This theoretical gap generates four concrete problems:

\begin{itemize}
	\item \textbf{Inconsistent implementation}: Without clear guiding principles, pedagogical innovations are applied superficially and often contradictorily
	\item \textbf{Systemic resistance}: The absence of a robust theory facilitates regression toward traditional models during moments of institutional pressure
	\item \textbf{Limited scalability}: Best practices do not scale effectively without theoretical understanding of underlying mechanisms
	\item \textbf{Inadequate evaluation}: The absence of theoretically grounded metrics prevents rigorous evaluation of innovative interventions
\end{itemize}

\subsection{The AI integration gap}

The second critical gap concerns the integration of Artificial Intelligence into educational processes. While the literature abounds with AI applications for individual learning personalization \cite{Holmes2023}, the potential of AI as a facilitator of collaborative cognitive processes and mediator of constructive epistemic conflicts remains largely unexplored.

Most current implementations of educational AI are based on obsolete behaviorist paradigms:
\begin{itemize}
	\item \textbf{Tutoring systems}: Digitally replicate the traditional transmissive model
	\item \textbf{Adaptive learning}: Personalize the path but maintain cognitive isolation
	\item \textbf{Automated assessment}: Automate evaluation without rethinking its foundations
\end{itemize}

What is missing is a vision of AI as a \textit{cognitive amplifier} for collective intelligence, a theoretical void that this thesis intends to fill by introducing the concept of \textit{non-agentive algorithmic facilitation}.

\section{The Pyragogical proposal: a systemic response}

\subsection*{Genesis and epistemological foundations:}

The term ``Pyragogy'' emerges from the confluence of three intellectual traditions previously considered incompatible: post-Darwinian evolutionary biology, post-Freirian critical pedagogy, and computational complexity science. This synthesis does not represent a mere interdisciplinary exercise, but constitutes what Thomas Kuhn would define as a ``paradigm shift'' in educational epistemology \cite{Kuhn1962}.

Pyragogy is rooted in a radical but rigorously founded premise: knowledge is not an individual property to be accumulated competitively, but an emergent phenomenon that evolves through natural selection processes applied to the cognitive domain \cite{Dennett1995}. In this framework, \textit{ideas} -- not individuals -- constitute the fundamental units subjected to selective pressure, variation, and adaptation.

\subsection{Operational definition:}

\begin{definition}[Pyragogy]
	\label{def:pyragogy}
	\textit{Pyragogy} is an adaptive and complex educational system, characterized by three tightly integrated components:
	\begin{enumerate}
		\item \textbf{Cognitive intraspecific selection}: evolutionary processes through which ideas and epistemic constructs compete, combine, and transform within learning ecosystems.
		\item \textbf{Epistemic reciprocation}: mutualistic interaction structures in which knowledge generation and reception constitute inseparable co-evolutionary processes.
		\item \textbf{Non-agentive algorithmic facilitation}: employment of artificial intelligence as a procedural amplifier, devoid of autonomous epistemic agency, that supports growth and reflection without replacing human thought.
	\end{enumerate}
	This definition deliberately distinguishes itself from previous conceptions of collaborative learning, introducing the concept of \textit{epistemic fitness}: the capacity of an idea to survive, replicate, mutate, and adapt through multiple minds and diversified contexts.
\end{definition}

\subsection{Transposition of the selection principle:}

The transposition of the natural selection principle from the biological to the cognitive domain requires a systematic mapping of structural correspondences. In Pyragogy, this mapping is articulated through four fundamental isomorphisms validated by scientific literature:

\textbf{Variation $\rightarrow$ Epistemic diversity}: As genetic mutations introduce variability in biological populations, diversity of cognitive perspectives generates variations in the available pool of ideas. Page's research \cite{Page2007} mathematically demonstrates that cognitively diverse groups systematically outperform homogeneous groups of more ``intelligent'' individuals in solving complex problems.

\textbf{Selection $\rightarrow$ Argumentative pressure}: Ideas are subjected to ``selective pressure'' through critical confrontation, empirical verification, and logical coherence. The argumentative theory of reason by Mercier and Sperber \cite{Mercier2017} provides the neurocognitive foundations for this process, demonstrating that human reasoning evolved primarily for the evaluation of arguments in social contexts.

\textbf{Heritability $\rightarrow$ Cultural transmission}: The mechanisms of cultural transmission described by the dual inheritance theory of Boyd and Richerson \cite{Boyd2005} provide the analogue of genetic inheritance. ``Surviving'' ideas are encoded in the group's collective memory through processes of cultural institutionalization.

\textbf{Adaptation $\rightarrow$ Conceptual refinement}: Ideas evolve by adapting to the ``epistemic landscape'' -- the multidimensional environment of problems, constraints, and cognitive opportunities that the group faces. Kauffman's theory of fitness landscapes \cite{Kauffman1993} provides the mathematical framework for understanding this dynamic.

\subsection{The transformative role of Cognitive Reciprocation}

Central to the Pyragogical model is the principle of \textit{Cognitive Reciprocation} (CR), mathematically formalized through the equation:

\begin{equation}
	CR = \frac{\sum_{i,j} \beta_{ij} \cdot V_{ij}}{\sum_{i} V_{i,in} + \sum_{j} V_{j,out}}
	\label{eq:reciprocazione}
\end{equation}

where:
\begin{itemize}
	\item $\beta_{ij}$ represents the bidirectional transformation coefficient between contribution $i$ and reception $j$
	\item $V_{ij}$ denotes the epistemic value of the exchange
	\item $V_{i,in}$ and $V_{j,out}$ normalize with respect to the total volume of exchanges
\end{itemize}

This formalization, derived from Nowak's evolutionary game theory \cite{Nowak2006}, captures the intuition that in an optimal pyragogical system, every act of teaching is simultaneously an act of learning. This is not about educational altruism but \textit{cognitive mutualism}: the benefit to the recipient amplifies the benefit to the donor through positive feedback mechanisms.

\newpage
\section{Thesis objectives and contributions}
\subsection*{Theoretical objectives:}

This thesis pursues four interconnected theoretical objectives:

\begin{enumerate}
	\item \textbf{Epistemological systematization}: Develop a rigorous theory of cognitive intraspecific selection, providing for the first time a unifying conceptual framework for the transposition of evolutionary principles to educational processes
	
	\item \textbf{Formalization of Reciprocation}: Mathematically define the operational mechanisms of epistemic reciprocation, specifying how bidirectional exchange dynamics can be optimized to maximize ecosystemic learning
	
	\item \textbf{Theory of educational AI}: Conceptualize a new paradigm for the role of artificial intelligence in collaborative educational processes, surpassing both anthropomorphization and technological instrumentalization
	
	\item \textbf{Multidisciplinary integration}: Synthesize neuroscientific, pedagogical, computational, and philosophical perspectives into a coherent framework for 21st century education
\end{enumerate}

\subsection{Empirical objectives:}

On the operational level, the research aims to:

\begin{enumerate}
	\item \textbf{Experimental validation}: Design and implement the \textit{IdeoEvo} pilot project, a controlled environment for testing the effectiveness of Pyragogy in authentic educational contexts
	
	\item \textbf{Metric development}: Develop and validate innovative metrics for evaluating ecosystemic learning, including the Epistemic Quality Index (EQI) and complementary metrics
	
	\item \textbf{Replicable protocols}: Document standardized protocols for implementing Pyragogy in different types of educational institutions, ensuring scalability and contextual adaptability
\end{enumerate}

\subsection{Expected contributions:}

The work aspires to generate contributions distributed across three levels:

\textbf{Theoretical level}:
\begin{itemize}
	\item First rigorous systematization of cognitive intraspecific selection as an educational framework
	\item Mathematically grounded formal model of Cognitive Reciprocation
	\item Theory of symbiotic AI-human integration in collaborative cognitive processes
	\item Epistemological framework for evaluating ideas independently of their individual bearers
\end{itemize}

\textbf{Methodological level}:
\begin{itemize}
	\item Innovative protocols for productive ritualization of cognitive conflicts
	\item Validated tools for evaluating ecosystemic learning
	\item Operational framework for mitigating algorithmic bias in education
	\item Methodologies for designing evolutionary-adaptive learning environments
\end{itemize}

\textbf{Practical level}:
\begin{itemize}
	\item Empirically tested implementation model for educational institutions
	\item Structured curriculum for specialized training of pyragogical educators
	\item Evidence-based policy recommendations for systemic innovation in education
	\item Open-source technological platform for implementing pyragogical tools
\end{itemize}

\section{Structure and organization of the work}

The thesis is organized to guide the reader through a logical path from theoretical foundation to practical application:

\textbf{Chapter 2} -- \textit{Theoretical reference framework}: Presents the multidisciplinary scientific foundations of the research, with particular focus on intraspecific selection in biology, its traditional pedagogical transpositions, contemporary cooperative models, and the origins of Pyragogy in the Peeragogy movement.

\textbf{Chapter 3} -- \textit{The Pyragogical Model}: Systematically defines operational principles, activation mechanisms, and conceptual architecture of the pyragogical framework, with particular attention to the formalization of Cognitive Reciprocation.

\textbf{Chapter 4} -- \textit{Evaluation Metrics}: Introduces and rigorously defines the Epistemic Quality Index (EQI) and complementary metrics, together with evaluation protocols and monitoring tools.

\textbf{Chapter 5} -- \textit{Experimental Design: IdeoEvo Project}: Describes in detail the pilot project for empirical validation of the model, including objectives, hypotheses, methodology, timeline, and ethical considerations.

\textbf{Chapter 6} -- \textit{Discussion}: Explores the educational, social, neurocognitive, and epistemological implications of the proposed model, analyzing limits, challenges, and future directions.

\textbf{Chapter 7} -- \textit{Conclusions}: Synthesizes the main contributions, traces development prospects, and outlines the transformative vision of Pyragogy for future education.

\textbf{Appendix A} -- \textit{Critical Issues and Implementation Solutions}: Systematically addresses the main practical challenges of pyragogical implementation, proposing concrete and realistic solutions for the transition from traditional models.

\textbf{Appendix B} -- \textit{Mathematical Formalization}: Provides a rigorous mathematical treatment of the key concepts of Pyragogy, including the formalization of Cognitive Reciprocation, derivation of Epistemic Fitness metrics, and dynamic models simulating cognitive intraspecific selection cycles.

\textbf{Appendix C} -- \textit{AI Study Prompt}: Presents a detailed prompt and methodology for studying and deepening Pyragogy with AI support, including step-by-step instructions for analyzing chapters, generating examples, applying metrics, simulating conceptual experiments, and translating or elaborating text in LaTeX while correcting errors and maintaining consistency.


\section{Methodological note}

This study adopts a \textit{design-based research} methodological approach, characterized by systematic integration of theory and practice through iterative cycles of design, implementation, evaluation, and refinement. The research is positioned at the intersection between educational sciences, computer science, cognitive neuroscience, and complexity theory, requiring an intrinsically interdisciplinary methodological approach.

The epistemological framework adopted is that of \textit{critical realism} \cite{Bhaskar2008}: we recognize the existence of real structures and mechanisms independent of our observation (realist ontology), but accept that our knowledge of such structures is always mediated and fallible (relativist epistemology). This positioning is particularly appropriate for the study of complex educational systems, where the interaction between human and technological components generates emergent properties not predictable from the simple sum of parts.

The thesis concludes with an invitation to transformation: it is not simply about proposing a new pedagogical method, but about radically rethinking the epistemological foundations of formal education, transforming competition from a destructive force into an evolutionary catalyst for human collective intelligence.