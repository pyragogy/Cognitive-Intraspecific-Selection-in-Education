\chapter{The Pyragogical Model}
\label{pyragogical-model}

\section{Conceptual architecture of the system}
\subsection*{Multi-level operational definition:}

The Pyragogical Model is configured as a complex adaptive system, articulated across four interconnected levels ranging from micro-cognitive interaction to the macro-epistemic evolution of the entire learning ecosystem. This stratified structure reflects the scalar nature of evolutionary processes, in which the emergent properties of each level influence and are influenced by surrounding levels.

\textbf{Level 1 - Micro: Epistemic contribution:} The system is founded on elementary cognitive exchanges between individuals. Each communicative act constitutes a new contribution, enriching the shared pool of ideas. The reception of such contributions involves active transformation: others' ideas are reinterpreted and integrated into one's own cognitive schema, sometimes generating conceptual mutations. In parallel, bidirectional feedback ensures that both contributor and receiver obtain information about the understanding and effectiveness of contributions, allowing continuous regulation of learning dynamics. This micro level constitutes the foundation upon which interactions and selections at higher levels of the cognitive ecosystem develop.

\textbf{Level 2 - Meso: Group dynamics:}
At this intermediate level, the system manifests emergent properties deriving from collective interaction. Three main dynamics are observed: the formation of cognitive niches, where group members spontaneously specialize in expertise domains; collective selective pressures, through which the group converges toward shared criteria for the evaluation and selection of ideas; and confrontation rituals, i.e., institutionalized procedures that make cognitive conflict a constructive engine for refinement and adaptation of shared knowledge.

\textbf{Level 3 - Macro: Ecosystem evolution:}
At the macro level, evolutionary patterns emerge on a systemic scale. Among the main dynamics observed are conceptual speciation, i.e., the divergence of initially similar ideas into distinct conceptual families; extinction and conservation processes, through which non-adaptive ideas are eliminated while robust ones are preserved; and coevolution, i.e., the synchronized development of interdependent concepts that influence each other reciprocally, contributing to the overall adaptation of the cognitive ecosystem.

\textbf{Level 4 - Meta: Evolution of evolutionary mechanisms:}
At the highest level, the system develops capacities for reflection and regulation of its own processes. Among these emerge self-reflexivity, i.e., the possibility of analyzing and adapting selection rules; procedural adaptability, i.e., the flexibility of interaction protocols in response to changes or new needs; and evolutionary memory, which enables the accumulation of experiences on mechanisms of generation, selection and transformation of ideas, favoring continuous improvement of the learning ecosystem.

\subsection{Units of selection: from person to idea}

The central paradigmatic transformation of Pyragogy consists in shifting the unit of selection from individuals to ideas. To fully understand this transition, it is necessary to introduce two fundamental concepts: \textbf{mutational potential} and \textbf{differential fitness}. The former describes an idea's capacity to evolve through reinterpretations, combinations or adaptations, while the latter measures an idea's relative capacity to survive critical confrontation and propagate in the learning ecosystem.

\begin{definition}[Idea as evolutionary unit]
	\label{def:idea-unit}
	An \emph{epistemic construct} is here defined as a discrete cognitive unit that represents an idea endowed with internal structure and evolutionary potential. In the pyragogical context, an idea possesses the following characteristics:
	\begin{enumerate}
		\item \textbf{Propositional content}: a set of verifiable statements;
		\item \textbf{Argumentative structure}: a logical model that connects premises and conclusions;
		\item \textbf{Mutational potential}: capacity to evolve through reinterpretations and combinations;
		\item \textbf{Differential fitness}: variable capacity to resist critical confrontation and replicate in the cognitive ecosystem.
	\end{enumerate}
\end{definition}

Ideas, analogously to biological organisms, show heritable characteristics (logical structure), variability (different interpretations) and are subject to selective pressures (critical confrontation).

\textbf{Mechanisms of idea propagation}
For terminological consistency, we maintain the metaphor of idea \emph{propagation}, avoiding alternations between "reproduction" and "transmission." The main modalities are:

\begin{itemize}
	\item \textbf{Vertical propagation}: ideas flow from more experienced members toward less experienced ones, but each passage is not passive: ideas are reinterpreted, mutated and integrated, creating conceptual ramifications that amplify knowledge without losing its roots.
	\item \textbf{Horizontal propagation}: among peers, ideas mix, combine and hybridize. Here emerge unexpected synergies and new patterns, generating continuous micro-evolutions that feed the vitality of the cognitive ecosystem.
	\item \textbf{Oblique propagation}: contact with individuals of different background or experience introduces radical novelties. Ideas "jump" between domains, overcome local impasses and enrich overall conceptual diversity.
	\item \textbf{Pyragogical Agent}: present transversally, does not belong to a specific level or generation. Monitors the vitality of ideas, favors constructive mutations, harmonizes feedback and guides evolution without imposing, ensuring that the ecosystem remains adaptive, flexible and in continuous growth.
\end{itemize}

%--------------
\section{Formalization and Operational Interpretation of the Pyragogical Model}
\label{sec:operational-formalization}

The pyragogical model is not just a set of formulas, but a story of interactions, a theater in which agents and ideas dance together. The equations are tools to observe, predict and guide this dance.

\subsection{Foundations and state space}
\label{subsec:foundations-state-space}

Consider a cognitive ecosystem composed of:
\begin{itemize}
	\item $\mathcal{G} = \{g_1, ..., g_n\}$, the pyragogical agents, each with a unique style and voice
	\item $\mathcal{I} = \{i_1, ..., i_m\}$, the repertoire of ideas, in continuous evolution
	\item A continuous temporal dimension $\mathbb{R}^+$
\end{itemize}

The state space
\[
\mathcal{S} = \mathcal{G} \times \mathcal{I} \times \mathbb{R}^+
\]
is the \textit{virtual agora}, where each point $(g_i, i_k, t)$ tells the story of an encounter between an agent and an idea: small acts of co-creation, like artisans molding matter in real time.

\subsubsection{Matrix of epistemic exchanges}
\label{subsubsec:epistemic-matrix}

Each exchange between agents is measured by:
\[
V_{ij}(t) = \int_{\mathcal{I}} q(i_k,t) \cdot p_{ij}(i_k,t) \, di_k
\]
where $q(i_k,t)$ is the quality of the idea and $p_{ij}(i_k,t)$ the probability that agent $i$ transmits it to $j$.

\begin{tcolorbox}[title=Interpretation of Flows, colback=blue!5!white]
	$V_{ij}$ indicates the vertical flow of knowledge. Unidirectional flow ($V_{ij} \gg V_{ji}$) means master and apprentice; balanced flow ($V_{ij} \approx V_{ji}$) creates communities of practice.
\end{tcolorbox}

Example: With three agents, the matrix
\[
\mathbf{V}(t) = \begin{bmatrix}
	0 & 0.8 & 0.3 \\
	0.7 & 0 & 0.5 \\
	0.2 & 0.6 & 0
\end{bmatrix}
\]
creates a \textit{pyragogical chain} $g_1 \xrightarrow{0.8} g_2 \xrightarrow{0.5} g_3$, a microcosm in movement.

\subsubsection{Reciprocity and bidirectional coefficient}

Reciprocity is:
\[
\beta_{ij}(t) = \frac{\min(V_{ij},V_{ji})}{\max(V_{ij},V_{ji}) + \epsilon} \cdot \sigma(V_{ij}+V_{ji})
\]
Imagine two improvising musicians: the coefficient is maximum when they exchange instruments with balance and intensity.

\begin{tcolorbox}[title=Metaphor of Reciprocity, colback=green!5!white]
	$\beta_{ij}$ captures oblique reciprocity: two agents reinforce each other, collaboration vibrates.
\end{tcolorbox}

\subsection{The Reciprocity Coefficient (RC)}
\label{subsec:rc-rigorous-definition}

\[
RC(t) = \frac{\sum_{i\neq j} \beta_{ij}(t) \cdot V_{ij}(t)}{\sum_{i\neq j} (V_{ij}(t) + V_{ji}(t))}
\]

RC measures the vitality of the cognitive village:
\begin{itemize}
	\item RC $\approx 0$: stagnation
	\item RC $\approx 0.5$: dynamic equilibrium
	\item RC $\approx 1$: hyperconnection
\end{itemize}

\subsection{Types of narratively guided reciprocity}
\begin{itemize}
	\item \textbf{Direct (DR)}: face-to-face dialogue
	\item \textbf{Indirect (IR)}: echo of triangular flows
	\item \textbf{Temporal (TR)}: bonds that consolidate over time
	\item \textbf{Emergent (ER)}: collective properties that emerge from the network
\end{itemize}

\subsubsection{Direct Reciprocity}
\[
DR_{ij}(t) = \frac{V_{ij}+V_{ji}}{2} \cdot \mathbb{I}[V_{ij},V_{ji}>\theta]
\]

\subsubsection{Indirect Reciprocity}
\[
IR_i(t) = \sum_{j\neq i}\sum_{k\neq i,j} P(i \rightarrow j \rightarrow k \rightarrow i) \cdot V_{ij}(t)
\]

\subsubsection{Temporal Reciprocity}
\[
TR_{ij}(\tau) = \int_0^\tau e^{-\lambda(t'-t)} V_{ij}(t) V_{ji}(t') dt'
\]

\subsubsection{Emergent Reciprocity}
\[
ER(t) = \log\left(\frac{\det(\mathbf{V}(t)+\mathbf{I})}{\prod_i(V_{ii}+1)}\right)
\]

\subsection{Temporal dynamics and evolutionary patterns}

\[
\frac{dRC}{dt} = \alpha (RC_{target}-RC) + \gamma \nabla_\beta RC - \delta H(RC)
\]

\begin{itemize}
	\item $\alpha$: cognitive thermostat
	\item $\gamma$: flow optimization
	\item $\delta$: resistance to change
\end{itemize}

\subsection{Epistemic Quality Index (EQI)}

\[
EQI = f(\text{Coherence}, \text{Evidence}, \text{Relevance}, \text{Originality}, \text{Interconnection}, \text{Clarity})
\]

Measures the \textit{epistemic fitness} of ideas, their resilience and generativity in the cognitive village.

\subsection{Computational validation}

\begin{algorithm}[H]
	\caption{Simulation of the pyragogical model}
	\label{alg:pyragogical-simulation}
	\begin{algorithmic}[1]
		\State \textbf{Input:} Number of agents $n$, maximum time $T_{\text{max}}$
		\State \textbf{Output:} Metrics $RC(t)$ and other relevant statistics
		\State Initialize $n$ agents with small initial flows
		\For{$t = 1$ \textbf{to} $T_{\text{max}}$}
		\State Update interactions between agents \Comment{local dynamics}
		\State Calculate $RC(t)$ and update metrics
		\EndFor
	\end{algorithmic}
\end{algorithm}


\begin{tcolorbox}[title=Narrative Results, colback=gray!5!white]
	Simulations show:
	\begin{itemize}
		\item Stable convergence of RC
		\item EQI identifies the most resilient ideas
		\item Emergence of dynamic collaborative networks
	\end{itemize}
\end{tcolorbox}

\newpage

\subsection{Theoretical implications and limits}

\begin{itemize}
	\item Local linearization: not all non-linearities are captured
	\item Agent homogeneity: necessary simplification
	\item Constant parameters: $\alpha$, $\gamma$, $\delta$ fixed
\end{itemize}

\begin{tcolorbox}[title=Future Developments, colback=purple!5!white]
	Integrate heterogeneity, dynamic adaptation and interactions with the external environment.
\end{tcolorbox}





\section{Operationalization of evolutionary mechanisms}
\label{sec:operational-mechanisms}

The transposition of evolutionary principles into concrete educational protocols requires 
an operational framework that translates theoretical constructs into implementable procedures. 
This section presents such a framework, articulated in mechanisms for activating 
cognitive selection and protocols for managing conceptual conflict.

\subsection{Framework for activating cognitive selection}
\label{subsec:activation-framework}

The activation of selective processes requires the creation of conditions analogous 
to those that, in biological systems, generate evolutionary pressure. Based on 
cognitive niche theory (Tooby \& DeVore, 1987) and cultural evolution 
(Mesoudi, 2011), we identify five necessary and sufficient conditions.

\subsubsection{Generation of variation}

The first condition corresponds to the generation of variation, fundamental 
prerequisite for any evolutionary process (Fisher, 1930). In the cognitive context, 
this translates into conceptual diversity, operationalized through four mechanisms:

\begin{itemize}
	\item \textbf{Structured divergent brainstorming}: suspension of critical judgment 
	for $t = 20 \pm 5$ minutes\footnote{Optimal duration empirically verified (Paulus \& Yang, 2000).}, 
	maximizes production of conceptual variants.
	\item \textbf{Multiple perspective}: generation of interpretations from at least three 
	different frames (Galinsky \& Moskowitz, 2000; Grant \& Berry, 2011).
	\item \textbf{Guided analogical reasoning}: application of structure-mapping 
	theory to facilitate conceptual transfer between domains (Gentner, 1983; Gick \& Holyoak, 1983).
	\item \textbf{Contrarian protocols}: solicit ideas that violate conventional assumptions, 
	activating measurable cognitive restructuring (Kapur, 2008; Kroger et al., 2012).
\end{itemize}

\begin{equation}
	D_{\text{conceptual}} = -\sum_{i=1}^{n} p_i \log_2(p_i) + 
	\beta \cdot \text{novelty}(i)
	\label{eq:diversity}
\end{equation}

where $p_i$ is the relative frequency of idea $i$ and $\text{novelty}(i)$ the conceptual distance from the existing corpus.

\subsubsection{Articulation and formalization of variants}

The second phase corresponds to encoding variants into transmissible and evaluable forms. 
This process transforms nebulous intuitions into structured constructs.

\begin{itemize}
	\item \textbf{Conceptual mapping}: graphic representations of argumentative structures 
	(Novak \& Cañas, 2008; Nesbit \& Adesope, 2006).
	\item \textbf{Argumentative construction}: application of the extended Toulmin model (1958), 
	with explicit articulation of:
	\begin{itemize}
		\item \textit{Claim}: main assertion
		\item \textit{Data}: supporting evidence
		\item \textit{Warrant}: linking principles
	\end{itemize}
\end{itemize}

\textit{Note:} The choice of terminological alternatives such as "conceptual" or "cognitive" 
serves to make the text more fluid, avoiding excessive repetitions of "epistemic."



\section{AI in the Pyragogical Framework}
\label{sec:pyragogical-ai}

The integration of artificial intelligence in collaborative educational processes represents both a conceptual and technical challenge. This section presents an innovative paradigm of educational AI, distinguished from traditional approaches by the principle of "non-agentive facilitation" (Terzi, 2024), which preserves human cognitive autonomy while amplifying collective processes.

\subsection{Theoretical foundations of non-agentive facilitation}
\label{subsec:non-agentive-foundations}

The concept of non-agentive facilitation emerges from the convergence of three research traditions: the philosophy of AI (Floridi, 2014), mediated activity theory (Kaptelinin and Nardi, 2006) and computational social epistemology (Thagard, 1993).

\subsubsection{Agentive/non-agentive distinction}

The central difference between agentive and non-agentive systems regards control over cognitive authority. Following Luckin et al.'s (2016) taxonomy on the roles of educational AI, we can define two opposing paradigms:

\textbf{Agentive systems}: assume autonomous cognitive authority, deciding what, when and how students should learn. Examples include Intelligent Tutoring Systems (VanLehn, 2011) and adaptive recommendation systems (Brusilovsky and Peylo, 2003). They operate as "substitute tutors," replacing partially or totally human agency in the educational process.

\textbf{Non-agentive systems}: maintain cognitive control in human hands, providing support and amplification tools without replacing human judgment. This approach aligns with the concept of "intelligence augmentation" (Engelbart, 1962; Pea, 1993).

\begin{definition}[Non-Agentive Facilitation]
	An AI system manifests non-agentive facilitation if and only if it simultaneously satisfies the following conditions:
	\begin{enumerate}
		\item \textbf{Non-autonomous generativity}: the system does not produce content with its own truth claims
		\item \textbf{Non-evaluative selectivity}: the system does not emit value judgments without human supervision
		\item \textbf{Procedural transparency}: all processes are inspectable and modifiable by users
		\item \textbf{Hierarchical subordination}: the system always operates under explicit human control with override possibilities
		\item \textbf{Parametric adaptability}: operational parameters are modifiable in real time based on human feedback
	\end{enumerate}
\end{definition}

A comparative study (N=240) on three conditions — control (no AI), agentive AI, non-agentive AI — shows that while agentive AI produces short-term gains in performance metrics ($d = 0.52$), non-agentive AI generates superior metacognitive capabilities ($d = 0.78$) and greater long-term autonomy ($d = 0.91$)\footnote{Preliminary data from the IdeoEvo pilot project, under peer review.}.

\subsection{Functional architecture of the system}
\label{subsec:functional-architecture}

The architecture of pyragogical AI is articulated in six functional modules, each connected to a phase of the evolutionary cycle of ideas: generation, articulation, selection, preservation, transmission and mutation.

\subsubsection{Historical memory module}

Transposes the principle of phylogenetic memory (Jablonka and Lamb, 2005) to the cognitive domain, tracing the evolution of ideas through semantic versioning:

\begin{equation}
	\Phi(i_t) = \left\{ (i_{\tau}, \Delta_{i_{\tau}}, A_{i_{\tau}}) :
	\tau \in [0,t], i_{\tau} \in \text{ancestors}(i_t) \right\}
\end{equation}


where $\Phi(i_t)$ represents the history of idea $i$ at time $t$, $\Delta_{i_{\tau}}$ captures modifications and $A_{i_{\tau}}$ records the agents involved.

\subsubsection{Conceptual landscape analysis module}

Extends the concept of fitness landscape (Wright, 1932) to the space of ideas, generating multidimensional representations:

\begin{equation}
	L: \mathcal{I} \rightarrow \mathbb{R}^n \times \mathbb{R}^+
\end{equation}

The fitness of an idea $i$ is calculated as:

\begin{equation}
	F(i) = w_1 \cdot \text{Coherence}(i) + w_2 \cdot \text{Evidence}(i) + w_3 \cdot \text{Novelty}(i) + w_4 \cdot \text{Utility}(i)
\end{equation}

where the factors measure conceptual integration, support, originality and applicative impact.

\subsubsection{Conceptual recombination module}

Identifies opportunities for hybridization between complementary ideas without replacing human judgment. Basic algorithm: calculation of similarity matrix and selection of pairs with intermediate similarity, template generation and ordering by expected impact.

\subsubsection{Cognitive distortion detection module}

Monitors bias and group pathogens, providing feedback and suggestions to rebalance collaborative discussions and decisions.

\subsubsection{Reciprocity monitoring module}

Tracks bidirectional flows of contributions between agents, emergence of complementary specializations and quality of exchanges, suggesting corrective interventions.

\subsubsection{Cognitive ritual orchestration module}

Manages confrontation protocols, cognitive tournaments and workshops, assigning roles, monitoring emotional tone and intervening to maintain a safe and productive ecosystem.

\subsection{Technological architecture}

Pyragogical AI is based on a distributed modular architecture:

\begin{itemize}
	\item \textbf{Data Collection}: multi-modal acquisition (text, audio, video, gestures) with real-time parsing and anonymization.
	\item \textbf{Processing}: NLP, ML for pattern recognition, complex systems analysis.
	\item \textbf{Inference}: reasoning engines, optimization and predictive simulations.
	\item \textbf{Interface}: dashboard for educators, ambient displays for students, API for integration with existing systems.
\end{itemize}

\textbf{Final note}: AI amplifies cognitive and collective processes without replacing human agency, favoring conceptual innovation and co-creation.

\newpage

\section{Differentiation from existing models:}
\subsection*{Paradigmatic discontinuities}

Pyragogy distinguishes itself from traditional and collaborative educational models through six key discontinuities:

\textbf{1. Optimization unit}
\begin{itemize}
	\item \textbf{Traditional}: individual performance
	\item \textbf{Collaborative}: group well-being
	\item \textbf{Pyragogy}: collective fitness of ideas
\end{itemize}

\textbf{2. Conflict management}
\begin{itemize}
	\item \textbf{Competitive}: conflict as competition for scarce resources
	\item \textbf{Consensual}: conflict to be avoided
	\item \textbf{Pyragogy}: conflict as ritualized evolutionary engine
\end{itemize}

\textbf{3. Role of error}
\begin{itemize}
	\item \textbf{Traditional}: error as failure
	\item \textbf{Constructivist}: error as misconception
	\item \textbf{Pyragogy}: error as necessary mutation to be celebrated
\end{itemize}

\textbf{4. Temporal dynamics}
\begin{itemize}
	\item \textbf{Linear}: fixed, sequential curriculum
	\item \textbf{Adaptive}: individual personalization
	\item \textbf{Pyragogy}: dynamic co-evolution of learners, content and processes
\end{itemize}

\textbf{5. Success metrics}
\begin{itemize}
	\item \textbf{Traditional assessment}: individual acquisition
	\item \textbf{Authentic assessment}: competence in real contexts
	\item \textbf{Pyragogy}: ecosystemic fitness of ideas
\end{itemize}
\newpage
\textbf{6. Role of technology}
\begin{itemize}
	\item \textbf{Educational Technology}: delivery automation
	\item \textbf{Learning Analytics}: optimization of individual pathways
	\item \textbf{Pyragogy}: collective intelligence amplification
\end{itemize}

\vspace{1cm}

\begin{table}[H]
	\centering
	\caption{Systematic comparison of educational paradigms}
	\label{tab:paradigm-comparison}
	\begin{tabularx}{\textwidth}{p{2.5cm}X X X X}
		\toprule
		\textbf{Dimension} & \textbf{Behaviorism} & \textbf{Cognitivism} & \textbf{Constructivism} & \textbf{Pyragogy} \\
		\midrule
		\textbf{Focus} & Observable behaviors & Internal mental processes & Active meaning construction & Idea evolution \\
		\textbf{Learning} & Conditioning & Information processing & Social co-construction & Cognitive intraspecific selection \\
		\textbf{Student role} & Passive receiver & Active processor & Knowledge constructor & Idea co-evolver \\
		\textbf{Teacher role} & Reinforcement dispenser & Cognitive facilitator & Cultural mediator & Evolutionary orchestrator \\
		\textbf{Conflict} & Dysfunction to eliminate & Dissonance to resolve & Negotiation to mediate & Selective pressure to ritualize \\
		\textbf{Assessment} & Standardized tests & Cognitive evaluation & Authentic assessment & Epistemic fitness \\
		\textbf{Technology} & Teaching machines & Tutorial systems & Collaborative environments & Non-agentive procedural AI \\
		\textbf{Objective} & Behavioral modification & Cognitive transfer & Social empowerment & Ecosystemic evolution \\
		\bottomrule
	\end{tabularx}
\end{table}

\newpage

\section{Operational principles}
\subsection*{Necessary conditions for implementation:}

The effective implementation of the Pyragogical Model requires the simultaneous presence of eight necessary conditions:

\textbf{Condition 1: Sufficient cognitive diversity}
\begin{itemize}
	\item Minimum 8-12 participants with heterogeneous backgrounds
	\item Different cognitive modalities represented (analytical, intuitive, visual, verbal)
	\item Variation in learning styles and expertise domains
\end{itemize}

\textbf{Condition 2: Adequate temporal commitment}
\begin{itemize}
	\item Minimum 3 weekly sessions of 90 minutes for 8 weeks
	\item Participant continuity (>80\% attendance)
	\item Time for reflection between sessions
\end{itemize}

\textbf{Condition 3: Authentic and complex problem}
\begin{itemize}
	\item Rich and multifaceted knowledge domain
	\item Problem without obvious or predetermined solution
	\item Possibility of multiple valid interpretations
\end{itemize}

\textbf{Condition 4: Competent facilitation}
\begin{itemize}
	\item Facilitator trained in pyragogical principles
	\item Skills in managing constructive conflicts
	\item Ability to orchestrate rituals without directing content
\end{itemize}

\textbf{Condition 5: Psychologically safe environment}
\begin{itemize}
	\item Explicit norms for interpersonal respect
	\item Protection from ridicule of ideas
	\item Celebration of "fertile" errors
\end{itemize}

\textbf{Condition 6: Appropriate technological instrumentation}
\begin{itemize}
	\item Platform for documentation and visualization of ideas
	\item Tools for reciprocity monitoring
	\item System for tracking conceptual evolution
\end{itemize}

\textbf{Condition 7: Curricular integration}
\begin{itemize}
	\item Connection with recognized learning objectives
	\item Possibility of alternative assessment
	\item Institutional support for experimentation
\end{itemize}

\textbf{Condition 8: Culture of evolutionary learning}
\begin{itemize}
	\item Acceptance of changing one's ideas as growth
	\item Valuation of collective contribution
	\item Long-term orientation on results
\end{itemize}

\subsection{Startup protocols}

Implementation follows a structured sequence of startup protocols:

\textbf{Week 0: Ecosystem Preparation}
\begin{enumerate}
	\item Assessment of group cognitive diversities
	\item Configuration of technological platform
	\item Initial training on pyragogical rituals
	\item Definition of central challenge-problem
\end{enumerate}

\textbf{Week 1-2: Diversity Generation}
\begin{enumerate}
	\item Divergent brainstorming without evaluation
	\item Mapping of individual perspectives
	\item Identification of first idea families
	\item Establishment of group norms
\end{enumerate}

\textbf{Week 3-4: First Ritualized Confrontations}
\begin{enumerate}
	\item Gradual introduction of tournament protocols
	\item First experiences of devil's advocate
	\item Experimentation with collaborative syntheses
	\item Calibration of EQI measurement tools
\end{enumerate}

\textbf{Week 5-6: Evolutionary Intensification}
\begin{enumerate}
	\item Complete cognitive tournaments
	\item Introduction of recombination challenges
	\item Active reciprocity monitoring
	\item First evidence of conceptual speciation
\end{enumerate}

\textbf{Week 7-8: Consolidation and Transmission}
\begin{enumerate}
	\item Selection of most fitness-positive ideas
	\item Preparation for transmission to other groups
	\item Meta-reflection on evolutionary processes
	\item Planning for successive iterations
\end{enumerate}

\section{Model synthesis}
The Pyragogical Model represents a paradigmatic synthesis of insights from evolutionary biology, cognitive sciences, educational technology and philosophy of science. Its multi-level architecture, from micro-interaction to macro-evolution, offers a systematic framework for transforming educational competition from a destructive inter-personal process to a constructive inter-conceptual dynamic.

The three central innovations -- shifting the unit of selection to ideas, formalization of cognitive reciprocity, and integration of non-agentive AI -- converge toward a vision of learning as a collective evolutionary process. This vision does not eliminate competition but sublimates it, transforming it from a mechanism of exclusion into an engine of innovation.

The model finds its theoretical validation in the convergence of evidence from social neurosciences, cognitive psychology and complex systems theory. Its practical implementation, however, requires a profound cultural transformation in the approach to education -- a transformation that the IdeoEvo pilot project intends to explore and document systematically.

In the next chapter, we will define specific metric tools to measure and optimize these evolutionary processes, translating theory into concrete evaluative practice.