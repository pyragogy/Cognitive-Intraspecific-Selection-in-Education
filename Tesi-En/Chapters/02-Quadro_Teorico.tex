\chapter{Theoretical Framework}
\label{theoretical-framework}

\section{Intraspecific selection in evolutionary biology}
\subsection*{Darwinian foundations:}

The concept of intraspecific selection finds its theoretical roots in Charles Darwin's monumental work, \textit{On the Origin of Species by Means of Natural Selection} \cite{Darwin1859}, where it is defined as the process through which individuals of the same species compete for limited resources, generating selective pressures that favor specific adaptations. However, it is crucial to understand that Darwin conceived this competition not as a Hobbesian war of all against all, but as a complex process of ecological optimization.

Darwin's original formulation identified three fundamental components of intraspecific selection:

\begin{enumerate}
	\item \textbf{Variation}: The presence of heritable differences among individuals in the same population
	\item \textbf{Differential selection}: Unequal reproductive success based on specific characteristics
	\item \textbf{Heritability}: The transmission of advantageous characteristics to offspring
\end{enumerate}

Contemporary evolutionary research has significantly refined this understanding. Hamilton \cite{Hamilton1964} mathematically demonstrated how apparently altruistic behaviors can evolve through kin selection, while Trivers \cite{Trivers1971} formalized the theory of reciprocal altruism, showing how cooperation can emerge even among unrelated individuals.

\newpage

\subsection{Ritualization of conflict: Lorenz's insight}

Konrad Lorenz's ethology contributed extraordinarily to understanding intraspecific selection through the concept of \textit{conflict ritualization}. In his seminal work \textit{On Aggression} \cite{Lorenz1966}, Lorenz documents how many species have evolved behavioral mechanisms that channel intraspecific competition into non-lethal but functionally equivalent forms to direct competition.

\begin{example}[Ritualization in cichlid fish]
	\label{ex:cichlids}
	Ethological studies on cichlid fish (\textit{Cichlasoma}) show how males compete for territory through highly formalized ritual displays: coloration exhibitions, stereotyped movements and ``duels'' of increasing intensity that rarely result in actual physical damage. The ``winner'' obtains preferential access to resources without the ``loser'' being eliminated from the genetic pool.
\end{example}

This ritualization mechanism presents characteristics particularly relevant for pedagogical transposition:

\begin{itemize}
	\item \textbf{Diversity preservation}: Ritualized conflict does not eliminate ``losers,'' maintaining the genetic diversity necessary for future adaptability
	\item \textbf{Energy economy}: Energy spent in ritualized conflict is significantly lower than in lethal conflict
	\item \textbf{Social stability}: Ritualization produces stable hierarchies that reduce chronic conflict
	\item \textbf{Social learning}: Young individuals learn ``rituals'' by observing adults, creating cultural transmission
\end{itemize}

\subsection{Group selection and cooperation}

While classical natural selection theory focused on competition between individuals, contemporary research has rehabilitated the concept of group selection. Wilson and Wilson \cite{Wilson2007} have provided mathematical and empirical evidence that selection operates simultaneously at multiple levels:

\begin{equation}
	\Delta \bar{z} = \text{Cov}(w_i, z_i) + E[w_i \cdot \Delta z_i]
	\label{eq:multilevel-selection}
\end{equation}

where the first term represents selection between individuals and the second selection between groups.

This multi-level perspective is crucial for understanding how cooperative traits can evolve despite immediate individual disadvantages. Nowak's research \cite{Nowak2006} identifies five evolutionary mechanisms for cooperation:

\begin{enumerate}
	\item \textbf{Kin selection}: Cooperation toward genetically related individuals
	\item \textbf{Direct reciprocity}: Cooperation based on repeated interactions
	\item \textbf{Indirect reciprocity}: Cooperation mediated by reputation
	\item \textbf{Group selection}: Competitive advantages for cooperative groups
	\item \textbf{Network structure}: Cooperation facilitated by specific social topologies
\end{enumerate}

\section{Traditional pedagogical transpositions}
\subsection*{Successes and failures:}
\subsection{Educational social Darwinism}

The transposition of evolutionary principles to education has a long and problematic history. Herbert Spencer \cite{Spencer1864}, with his slogan ``survival of the fittest,'' inaugurated a tradition of social Darwinism that profoundly influenced Western educational systems. This perspective generated pedagogical practices characterized by:

\begin{itemize}
	\item \textbf{Zero-sum competition}: The conception of learning as a zero-sum game where some students' success necessarily implies others' failure.
	\item \textbf{Meritocratic selection}: The use of standardized tests and rankings to ``select'' the ``most fit and deserving.''
	\item \textbf{Elimination of the ``weak''}: Practices of exclusion and marginalization of students with lower performance.
\end{itemize}

\subsection{Sociological critique: Bourdieu and symbolic violence}

Pierre Bourdieu and Jean-Claude Passeron \cite{Bourdieu1977} provided a devastating systemic critique of educational social Darwinism through the concept of \textit{symbolic violence}. Their ethnographic and statistical research demonstrates how competitive educational systems do not actually select the ``most fit'' in a cognitive sense, but systematically reproduce pre-existing social inequalities.

\begin{table}[h]
	\centering
	\caption{Correlations between social origin and academic success in France (Bourdieu, 1977)}
	\label{tab:bourdieu-correlations}
	\begin{tabular}{lcc}
		\toprule
		\textbf{Social origin} & \textbf{University access (\%)} & \textbf{Elite degree (\%)} \\
		\midrule
		Working class & 12\% & 2\% \\
		Middle class & 34\% & 8\% \\
		Upper class & 78\% & 45\% \\
		\bottomrule
	\end{tabular}
\end{table}

Bourdieu identifies three mechanisms through which ``cultural capital'' is transformed into educational advantage:

\begin{enumerate}
	\item \textbf{Embodied cultural capital}: Durable dispositions, perceptual and categorical schemas acquired through primary socialization.
	\item \textbf{Objectified cultural capital}: Cultural goods (books, instruments, machines) that presuppose embodied capital to be utilized.
	\item \textbf{Institutionalized cultural capital}: Educational credentials that certify the possession of cultural capital.
\end{enumerate}

\subsection{The Deweyan alternative: democracy and experience}

John Dewey \cite{Dewey1916} proposed a radically different conception of education, based on principles of participatory democracy and experiential learning. His educational philosophy is founded on three fundamental principles:

\begin{itemize}
	\item \textbf{Learning by doing}: Learning through direct experience and solving authentic problems.
	\item \textbf{Continuity of experience}: Every experience modifies those who live it and influences the quality of subsequent experiences.  
	\item \textbf{Social interaction}: Learning as an intrinsically social and collaborative process.
\end{itemize}

Dewey anticipated as early as 1916 many of the principles we will find in Pyragogy:

\begin{quote}
	``\textit{A democratic society must, consistent with its ideal, allow intellectual participation of all its members in forming the values that regulate the group's life. This can happen only if all individuals have the opportunity to develop their distinctive capacities and discover the interests that will guide them toward their particular social function}'' \cite{Dewey1916}.
\end{quote}

\subsection{Vygotsky and the zone of proximal development}

Lev Vygotsky \cite{Vygotsky1978} enhanced understanding of learning through the concept of \textit{zone of proximal development} (ZPD), defined as ``the distance between the actual developmental level as determined by independent problem solving and the level of potential development as determined through problem solving under adult guidance or in collaboration with more capable peers.''

Vygotskian theory presents three fundamental insights for Pyragogy:

\begin{enumerate}
	\item \textbf{Social mediation}: Cognitive development is always mediated by social interaction and cultural tools.
	\item \textbf{Internalization}: Interpersonal processes gradually transform into intrapersonal processes.
	\item \textbf{Role of language}: Language is not only a communication tool but a thinking tool.
\end{enumerate}

\section{Contemporary cooperative models}
\subsection*{Successes and limitations:}
\subsection{Cooperative Learning: the Johnson systematization}

David W. Johnson and Roger T. Johnson \cite{Johnson1999} developed the most systematic framework for cooperative learning, identifying five essential elements:

\begin{enumerate}
	\item \textbf{Positive interdependence}: Students perceive they are linked in such a way that one cannot succeed unless all succeed.
	\item \textbf{Individual accountability}: Each student is responsible for their own learning and contributing to group success.  
	\item \textbf{Promotive face-to-face interaction}: Students help, support, and encourage each other.
	\item \textbf{Interpersonal and small group skills}: Students develop and use social skills necessary to work effectively together.
	\item \textbf{Group processing}: Groups periodically reflect on how they are working together and how to improve.
\end{enumerate}

The meta-analysis conducted by the Johnsons on over 900 studies reveals consistently positive effect sizes for cooperative learning:

\begin{table}[h]
	\centering
	\caption{Effect sizes of cooperative learning (Johnson \& Johnson, 2009)}
	\label{tab:cooperative-effect-sizes}
	\begin{tabular}{lcc}
		\toprule
		\textbf{Dependent variable} & \textbf{Effect size} & \textbf{N. studies} \\
		\midrule
		Academic achievement & 0.64 & 305 \\
		Knowledge retention & 0.70 & 180 \\
		Problem-solving accuracy & 0.93 & 129 \\
		Creativity & 0.42 & 67 \\
		Learning transfer & 0.58 & 89 \\
		\bottomrule
	\end{tabular}
\end{table}

\textbf{Limitations of the Johnson model}:
Despite documented effectiveness, traditional cooperative learning presents some significant limitations that Pyragogy intends to address:

\begin{itemize}
	\item \textbf{Absence of constructive conflict}: The model tends to minimize disagreement rather than channel it productively.
	\item \textbf{Focus on products rather than processes}: Attention remains on learning outcomes rather than the evolution of ideas and thinking.
	\item \textbf{Lack of selective mechanisms}: There is no systematic process to identify and reinforce the most promising arguments.
	\item \textbf{Static structure}: Groups and roles are typically fixed, limiting dynamic adaptability.
\end{itemize}

\subsection{Communities of Practice: Wenger's approach}

Etienne Wenger \cite{Wenger1998} introduced the concept of \textit{communities of practice}, defined as ``groups of people who share a concern or passion for something they do and learn how to do it better as they interact regularly.''

Communities of practice are characterized by three dimensions:

\begin{enumerate}
	\item \textbf{Domain}: A shared area of knowledge that defines the community's identity.
	\item \textbf{Community}: A group of people who interact and learn together.
	\item \textbf{Practice}: A shared repertoire of resources, experiences, and ways of addressing recurring problems.
\end{enumerate}

Wenger identifies four modes of belonging:

\begin{itemize}
	\item \textbf{Engagement}: Active participation in community practices.
	\item \textbf{Imagination}: Creation of images of the world and connections across time and~space.
	\item \textbf{Alignment}: Coordination of energies and activities to align with broader structures and processes.
	\item \textbf{Multi-membership}: Simultaneous membership in multiple communities. 
\end{itemize}

\vspace{1cm}

\textbf{Contributions of Communities of Practice to Pyragogy}:  
Wenger's model \cite{Wenger1998} offers relevant insights for Pyragogy, particularly:

\begin{itemize}
	\item It highlights the role of identity and participation in knowledge construction, showing how learning emerges from shared practice and social collaboration.
	\item It recognizes learning as a situated phenomenon, arising from interaction, negotiation, and co-learning among community members.
	\item It introduces the concept of \textit{legitimate peripheral participation}, describing how new members progressively access established practices and contribute to community dynamism.
\end{itemize}

\subsection{Limitations regarding Pyragogy}

Despite significant contributions, the traditional approach of Communities of Practice presents some limitations in the pyragogical context:

\begin{itemize}
	\item Absence of explicit mechanisms for selection, variation, and evolution of cognitive practices, central elements for the dynamics of Pyragogy interactions.
	\item Greater attention to individual professional identity rather than shared epistemic growth and collective adaptation.
	\item Lack of qualitative formalization of learning processes, limiting the possibility of modeling and simulation of emergent cognitive phenomena.
\end{itemize}

\footnote{The term \emph{Communities of Practice (CoP)} indicates groups of people who share knowledge and practices in a collaborative context.}

\newpage

\section{Peeragogy vs. Pyragogy}
\subsection*{The genesis of Pyragogy:}

The Peeragogy.org community was born in 2012 from the initiative of Howard Rheingold and a collective of international researchers \cite{Rheingold2012}. The term itself is a neologism combining ``peer'' and ``pedagogy,'' indicating a learning approach characterized by mutuality, self-organization, and knowledge co-production.

The fundamental principles of Peeragogy include:

\begin{itemize}
	\item \textbf{Horizontal learning}: Learning among peers rather than hierarchical
	\item \textbf{Distributed expertise}: Recognition that expertise is distributed in the community
	\item \textbf{Co-facilitation}: Shared facilitation of learning processes
	\item \textbf{Emergent curriculum}: Curriculum that emerges from group needs and interests
\end{itemize}

\subsection{The Peeragogy Handbook: collaborative evolution}

The \textit{Peeragogy Handbook} \cite{CorneliEtAl2016} represents a paradigmatic example of peer-to-peer knowledge production. Through three editions (2012, 2013, 2016), the handbook was written, revised, and refined by a global community of contributors using collaborative digital~tools.

The Handbook's structure reflects peeragogical principles:

\begin{enumerate}
	\item \textbf{Motivation}: Why people choose to learn together
	\item \textbf{Case Study}: Concrete examples of peeragogy in action
	\item \textbf{Patterns}: Recurring patterns in peer-to-peer learning  
	\item \textbf{Practice}: Operational strategies for implementing peeragogy
	\item \textbf{Technologies}: Digital tools to support collaboration
\end{enumerate}

\subsection{From self-organization to guided evolution}
While \textit{Peeragogy} values self-organization and horizontal knowledge sharing, \pyragogy{} introduces the principle of \textit{guided evolution}: ideas are not limited to emerging spontaneously, but are subjected to selection and transformation processes that orient their qualitative growth. This transition reflects a more mature understanding of evolutionary mechanisms: not every self-organization produces optimal outcomes, and forms of procedural guidance --- not directive but structuring --- can accelerate collective cognitive evolution.
	
\vspace{0.5cm}

\begin{table}[h]
	\centering
	\renewcommand{\arraystretch}{1.3} % aumenta spazio tra righe
	\setlength{\tabcolsep}{6pt}       % margini interni colonne
	\caption{Conceptual evolution from Peeragogy to Pyragogy}
	\label{tab:peeragogy-pyragogy-evolution}
	\begin{tabularx}{\textwidth}{>{\raggedright\arraybackslash}p{3.5cm} 
			>{\raggedright\arraybackslash}X 
			>{\raggedright\arraybackslash}X}
		\toprule
		\textbf{Dimension} & \textbf{Peeragogy} & \textbf{Pyragogy} \\
		\midrule
		Primary focus          & Knowledge distribution & Evolutionary dynamics of ideas \\
		Role of conflict       & To minimize through consensus & To ritualize for selection \\
		Technological mediation& Communication and collaboration tools & Procedural AI for evolutionary facilitation \\
		Reference theory       & Social constructivism and critical theory & Evolutionary biology and complexity science \\
		Selection mechanism    & Democratic consensus and self-selection & Natural selection applied to ideas \\
		Assessment             & Narrative peer assessment & Quantitative (IQE) + qualitative metrics \\
		Diversity management   & Inclusion and pluralism & Diversity as evolutionary engine \\
		\bottomrule
	\end{tabularx}
\end{table}

\section{Neural Foundations of Collaboration}
\subsection*{Social neuroscience of learning:}

Neuroscientific research of the last two decades has revealed that learning is an intrinsically social phenomenon at the neurological level. Functional neuroimaging studies show that collaborative learning activates specific neural circuits absent in individual learning.

\textbf{The mirror neuron system}: The discovery of mirror neurons by Rizzolatti and Craighero \cite{Rizzolatti2004} revolutionized understanding of social learning. These neurons activate both when an individual performs an action and when they observe the same action performed by others, providing the neurological basis for imitation and vicarious learning.

\textbf{Neural synchrony}: Studies by Dumas et al. \cite{Dumas2010} using dual-brain EEG show that during effective collaborative interactions, synchronization of neural oscillations occurs between participants' brains, particularly in alpha and gamma bands.

\textbf{Theory of mind and mentalizing network}: The brain network involved in ``theory of mind'' (ability to attribute mental states to others) activates intensely during collaborative learning, suggesting that understanding others' perspectives is central to knowledge co-construction processes \cite{Frith2012}.

\subsection{Neurobiology of cognitive conflict}

Neuroscientific research on cognitive conflict provides crucial empirical bases for Pyragogy. Neuroimaging studies show that cognitive conflict activates specific brain areas associated with learning and synaptic plasticity.

\textbf{Anterior Cingulate Cortex (ACC)}: The ACC activates when cognitive incongruencies are detected, functioning as an ``alert system'' that signals the need for higher-order cognitive processes \cite{Botvinick2004}.

\textbf{Prefrontal cortex}: Cognitive conflict activates prefrontal areas associated with executive control and working memory, promoting deeper elaboration processes \cite{Miller2001}.

\textbf{Conflict-induced neuroplasticity}: Research by Kounios and Beeman \cite{Kounios2014} demonstrates that experiencing cognitive conflict followed by resolution (insight) produces lasting changes in synaptic connectivity, particularly in the right hemisphere.

\subsection{Neural bases of cognitive reciprocity}

Neuroscientific studies on reciprocity and cooperation provide biological support for the Cognitive Reciprocation principle central to Pyragogy.

\textbf{Dopaminergic reward system}: Research by Rilling et al. \cite{Rilling2002} shows that acts of reciprocal cooperation activate the ventral dopaminergic system, the same circuit involved in primary rewards, suggesting that cognitive reciprocity can be intrinsically gratifying.

\textbf{Oxytocin and trust}: Studies by Kosfeld et al. \cite{Kosfeld2005} demonstrate that oxytocin, often called the ``trust hormone,'' facilitates cooperative behaviors and increases willingness to share knowledge.

\textbf{Neural basis of teaching}: Research by Straube et al. \cite{Straube2009} identifies specific neural circuits that activate during the act of teaching, distinct from those involved in learning, supporting the idea that teaching and learning are complementary but distinct processes.

\newpage
\section{The pyragogical perspective}
\subsection*{Identified theoretical gaps:}

Despite the richness of the examined literature, four significant theoretical gaps emerge that Pyragogy intends to fill:

\textbf{Gap 1: Absence of systematic evolutionary principles in education}
While educational models exist that use evolutionary metaphors, there is a lack of rigorous and systematic transposition of natural selection principles to learning processes.

\textbf{Gap 2: Lack of metrics for the ``fitness'' of ideas and concepts}
Traditional assessment systems measure individual performance but not the quality and adaptability of ideas themselves as independent entities.

\textbf{Gap 3: Superficial integration of AI in collaborative learning}
Current applications of educational AI remain anchored to individualist paradigms and do not explore AI's potential as a facilitator of collective intelligence.

\textbf{Gap 4: Absence of cognitive conflict ritualization}
While cognitive conflict is recognized as beneficial, systematic frameworks for its constructive management in educational contexts are lacking.

\subsection{Distinctive contribution of Pyragogy}

\pyragogy{} positions itself as an integrated response to these gaps, articulated through three fundamental directions:

\begin{enumerate}
	\item \textbf{Systematic transposition}: First formalization of intraspecific selection principles in the educational domain, with conceptual translation and theoretical foundation.
	
	\item \textbf{Innovative metrics}: Definition of two new metrics --- the Epistemic Quality Index (IQE) and the Reciprocation Coefficient (CR) --- that overcome the limits of conventional indicators.
	
	
	\item \textbf{Procedural AI}: Conceptualization of AI as a non-agentive facilitator of cognitive evolutionary processes.
	
	\item \textbf{Productive conflict}: Development of systematic practices that allow critical interactions among participants to stimulate the growth and maturation of ideas and concepts.
	
\end{enumerate}

\section{Synthesis and transition}

The examined theoretical framework shows progressive convergence toward more collaborative and socially structured educational approaches. From Bourdieusian critique of social Darwinism applied to education, through Dewey's alternative based on democratic education, to contemporary models of cooperative learning and Communities of Practice, a clear trajectory emerges aimed at overcoming destructive interpersonal competition.

However, this evolution remains incomplete. Current models, while effective, lack a unifying theory that explains the mechanisms of collaborative success and indicates how to systematically maximize their effects. Precisely in this void Pyragogy positions itself: not as rejection of competition, but as its transformation into an evolutionary and constructive process, oriented toward collective development of ideas.

The transition from the traditional competitive paradigm to the pyragogical one represents not a change of method but a change of educational ontology: from conception of knowledge as scarce private property to its reconceptualization as an emergent collective resource. In the next chapter, we explore how this ontological transformation translates into concrete operational mechanisms through systematic definition of the Pyragogical Model.